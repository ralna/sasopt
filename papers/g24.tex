% g24.tex
% Started: May 23rd, 2022
% This version: May 23rd, 2022

\documentclass[12pt,twoside]{article}

\usepackage{repdefs}
%\usepackage{/home/nimg/pCloudDrive/papers/repdefs}
\usepackage{subdepth}
\usepackage{amsmath}
\usepackage{mathtools}
\usepackage{hyperref}
\hypersetup{
    pdfborder={0 0 0},
    linkcolor={red!50!black},
    citecolor={blue!50!black},
    urlcolor={blue!80!black}
}

\newcommand{\alphack}{\alpha\s{C}_k}
\newcommand{\xck}{x\s{C}_k}
\newcommand{\sck}{s\s{C}_k}
\renewcommand{\half}{\frac{1}{2}}
\newcommand{\tmin}{t_{\min}}

\usepackage{subdepth}

\newcommand{\paperauthor}{Working note --- Nick et al.}
\newcommand{\papertitle}{Optimization methods for small angle scattering}

\title{\papertitle}
\author{\paperauthor}

\pagestyle{myheadings}
\markboth{Draft version - not for circulation - \today}{Working Note --- Optimization for SAS\hfill}
\newcommand{\theabstract}{{%\small
Notes
}}

\begin{document}

\maketitle

\section{Introduction}

We are interested in general polydisperse, multi-parameter small-angle
scattering (SAS) problems, but for simplicity we concentrate here on
the "cylinder" model. In this we seek the coordinates
(the length $l$ and radius $r$ of the cylinder, the angle $\theta$
between the cylinder axis and the beam, and the rotation $\phi$ of the
cylinder axis about beam) of a cylinder. We consider each in a distributional
sense, e.g., with $\omega^l(l)$ for which $\int \omega^l(l) dl = 1$,
and for each form a discrete approximation $\omega^l_i$, $i = 1, \ldots, n_l$
that satisfies $\sum_{i=1}^{n_l} \omega^l_i = 1$. We observe the 2D intensity image
$I(x,y)$ as a function of the scattering vectors $(x,y)$, and
again let $x_{i_x}$, $i_x = 1,\ldots n_x$, and
$y_{i_y}$, $i_y = 1,\ldots n_y$ be the $(x,y)$ coordinates of discrete
observation points. This leads to a continuous model\footnote{See
 \url{https://github.com/stfc-sciml/ffsas/blob/main/doc/USER-GUIDE.md}}
\disp{I(x,y,l,r,\theta,\phi) = \xi G(x,y,l,r,\theta,\phi) + \beta}
of the intensity involving a scaling factor $\xi$, background $\beta$ and
a suitable Green's function $G$, and thence to a discrete model
\disp{I_{i_x,i_y}(\omega^l, \omega^r, \omega^\theta, \omega^\phi, \xi, \beta) =
 \xi \sum_{i_l=1}^{n_l} \sum_{i_r=1}^{n_r} \sum_{i_\theta=1}^{n_\theta}
 \sum_{i_\phi=1}^{n_\phi} g_{i_x,i_y,i_l,i_r,i_{\theta},i_{\phi}}
\omega^l_{i_l} \omega^r_{i_r} \omega^\theta_{i_\theta} \omega^\phi_{i_\phi} + \beta}
involving the six-way tensor $g_{i_x,i_y,i_l,i_r,i_{\theta},i_{\phi}}$.
Given observations with means $\mu_{i_x,i_y}$ and standard deviations
$\sigma_{i_x,i_y}$ at the discrete observation points $(x_i,y_j)$, we then
endeavour to find vectors $\omega^l$, $\omega^r$, $\omega^\theta$, $\omega^\phi$
and scalars $\xi$ and $\beta$ by considering the $\chi^2$ error
\disp{ \chi^2(\omega^l, \omega^r, \omega^\theta, \omega^\phi, \xi, \beta)
 \define
\half \bigsum_{i_x=1}^{n_x} \bigsum_{i_y=1}^{n_y}
 \left( \frac{I_{i_x,i_y}(\omega^l, \omega^r, \omega^\theta, \omega^\phi, \xi, \beta)
 - \mu_{i_x,i_y}}{\sigma_{i_x,i_y}} \right)^2}
and solving the optimization problem
\eqn{sasopt}{\arr{rl}{
\minimizeover{\omega^l, \omega^r, \omega^\theta, \omega^\phi, \xi, \beta} &
\chi^2(\omega^l, \omega^r, \omega^\theta, \omega^\phi, \xi, \beta) \\
\mbox{subject to}
& \bigsum_{i_l=1}^{n_l} \omega^l_{i_l} = 1, \,
  \omega^l_{i_l} \geq 0,\; i_l = 1,\ldots,n_l, \\
& \bigsum_{i_r=1}^{n_r} \omega^r_{i_r} = 1, \,
  \omega^r_{i_r} \geq 0,\; i_r = 1,\ldots,n_r, \\
& \bigsum_{i_{\theta}=1}^{n_\theta} \omega^\theta_{i_\theta} = 1, \;
 \omega^\theta_{i_\theta} \geq 0,\; i_\theta = 1,\ldots,n_\theta, \\
& \bigsum_{i_{\phi}=1}^{n_\phi} \omega^\phi_{i_\phi} = 1, \;\mbox{and} \;
 \omega^\phi_{i_\phi} \geq 0,\; i_{\phi} = 1,\ldots,n_\phi.}}
Of note are that $I_{i,j}$ is a multilinear form of its parameters, and
that each set of constraints restricts the relevant variable set to
lie on its unit simplex
\disp{\Delta^n \define
 \{ x \st \sum_{i=1}^n x_i = 1, \; x_i \geq 0, i = 1, \ldots,n\}.}
Thus while \req{sasopt} is a nonlinear (nonconvex) optimization problem, it
would seem sensible to exploit the structure at hand.

Notice that if one of the vectors of variables
$\omega^l$, $\omega^r$, $\omega^\theta$, $\omega^\phi$, and $(\xi,\beta)$
is considered alone, while keeping the remainder fixed at feasible
values, we may improve the objective function by solving a {\em linear}
least-squares problem over the unit simplex. For example, if
$\omega^r$, $\omega^\theta$, $\omega^\phi$ and  $(\xi,\beta)$ are fixed at
feasible values, but the $\omega^l$ is allowed to vary, we may improve $\chi^2$
by solving the problem
\disp{\arr{rl}{
\minimizeover{\omega^l} & \half \bigsum_{i_x=1}^{n_x} \bigsum_{i_y=1}^{n_y}
 \left( \frac{ \bigsum_{i_l=1}^{n_l} g^l_{i_x,i_y,i_l} \omega^l_{i_l}
 - (\mu_{i_x,i_y}-\beta)}{\sigma_{i_x,i_y}} \right)^2 \\
\mbox{subject to}
& \bigsum_{i_l=1}^{n_l} \omega^l_{i_l} = 1, \;\mbox{and}\; \omega^l_{i_l} \geq 0,\;
 i_l = 1,\ldots,n_l,}}
where
\disp{ g^l_{i_x,i_y,i_l} \define \xi \sum_{i_r=1}^{n_r} \sum_{i_\theta=1}^{n_\theta}
 \sum_{i_\phi=1}^{n_\phi} g_{i_x,i_y,i_l,i_r,i_{\theta},i_{\phi}}
\omega^r_{i_r} \omega^\theta_{i_\theta} \omega^\phi_{i_\phi}.}
Likewise, if $\omega^l$, $\omega^r$, $\omega^\theta$ and $\omega^\phi$
are fixed but $(\xi,\beta)$ allowed to vary, we may again improve $\chi^2$ by
solving the (unconstrained, two variable) linear least-squares problem
\disp{\minimizeover{\xi,\beta} \;\half \bigsum_{i_x=1}^{n_x}
 \bigsum_{i_y=1}^{n_y}
 \left( \frac{ a_{i_x,i_y} \xi + \beta - \mu_{i_x,i_y}}{\sigma_{i_x,i_y}} \right)^2,}
where
\disp{a_{i,j} \define \sum_{i_l=1}^{n_l} \sum_{i_r=1}^{n_r} \sum_{i_\theta=1}^{n_\theta}
 \sum_{i_\phi=1}^{n_\phi} g_{i_x,i_y,i_l,i_r,i_{\theta},i_{\phi}}
\omega^l_{i_l} \omega^r_{i_r} \omega^\theta_{i_\theta} \omega^\phi_{i_\phi}.}
Thus a possibility is to cycle through the sets of variables
$\omega^l$, $\omega^r$, $\omega^\theta$, $\omega^\phi$ and  $(\xi,\beta)$
in some way, fixing all but one at the current best values
and minimizing with respect to the remainder.

\numsection{Linear least-squares over the unit simplex}

In this section we focus on the simplex-constrained linear least-squares
problem
\eqn{slls}{\minimizeover{x \in \Delta^n}
 f(x) \define \half \| A x - b \|^2,}
where the $m$ by $n$ matrix $A$ and $m$ vector $b$ are given.
Let $P(v)$ be the projection of a given $v$ into the simplex $\Delta^n$;
we will return shortly to see how this may be calculated. In addition, let
\eqn{active-set}{\calA(v) \define \{ 1 \leq i \leq n \st P(v)_i = 0\}}
be the set of indices of variables that are zero at $P(v)$. We let
$e$ be the vector of ones of appropriate length, and $s_{\calI}$ is the
vector whose components
are the components $[s]_i$ of a given vector $s$ for $i \in \calI$,
where $\calI \subseteq \calN \define \{1,\ldots,n\}$.
We will abuse notation slightly in what follows: a subscript $k$ will
refer to an iterate (and may be a vector or matrix), while other
subscripts $i$, $j$,
etc., will be components of vectors and higher-order tensors (if needed
the component of a vector iterate will have a subscript $i,k$).

\newpage

\subsection{A basic algorithm}

A very basic method to find an approximate solution method is as follows:

\algo{alg-slls-basic}{linear least-squares over the unit simplex}{
\begin{description}
\item[Step 0: Initialization.]
Given a convergence tolerance $\epsilon > 0$,
pick an initial estimate $x_0 \in \Delta^n$, and set a counter $k=0$.

\item[Step 1: Test for termination.]
Compute the residual $r_k = A x_k - b$ and gradient $g_k = A^T r_k$.
If
\disp{ \| P(x_k - g_k) - x_k \| \leq \epsilon,}
stop with the approximate solution $x_* \define x_k$.

\item[Step 2: Compute a Cauchy point.]
Compute the Cauchy direction
\eqn{cauchydir}{\sck = \argminover{s \in \smallRe^n} g_k^T s 
\;\;\mbox{subject to} \;\; e^T s = 0 \;\; \mbox{and}\;\; s \geq - x_k,}
the stepsize $\alphack>0$ that sufficiently reduces
$f(P(x_k + \alpha \sck))$, and the Cauchy point
$\xck = P(x_k + \alphack \sck)$.

\item[Step 3: Improve on the Cauchy point.]
If desired, compute an approximate solution $s_k$ to the
{\em equality-constrained} linear least-squares problem
\eqn{eslls}{\minimizeover{s \in \smallRe^n} \half \| A (\xck + s) - b\|^2
\; \mbox{subject to} \; e^T s = 0 \; \mbox{and} \; s_{\calA(\xck)} = 0.
}
Otherwise define the next iterate $x_{k+1} = \xck$ and go to Step 5.

\item[Step 4: Find the next iterate.]
Find a stepsize $\alpha_k>0$ that reduces
$f(P(\xck + \alpha s_k))$, and define the next iterate
$x_{k+1} = P(x_k + \alpha_k s_k)$.

\item[Step 5: Perform the next iteration.]
Increase $k$ by 1 and go to Step 1.
\end{description}
}

A few comments are in order. Firstly, in Steps 2 and 4 we need to consider
the least-squares objective function $f(x)$ along a piecewise arc
$x(\alpha;x,v) = P(x + \alpha v)$ emanating from a feasible point $x$
in the direction $v$ but bent so as to remain feasible with respect to
$\Delta^n$. We consider this further in due course. We have not specified
what ``approximate'' means here, but two possibilities are either to find
the exact minimizer along the arc, or to find a point that gives
``sufficient'' decrease to guarantee convergence.

Secondly, it is trivial to find the required Cauchy direction in 
\req{cauchydir}. For the problem in question is clearly a linear
program, and as the feasible set is bounded, its solution must occur 
at a vertex, that is a point
at which $n-1$ of the bounds $s \geq x_k$ are satisfied as equalities.
Suppose that variable $i$ is the exception. Then
\eqn{defs}{ s_j = - x_{j,k} \tim{for} j \neq i \tim{and} s_i = 1 - x_{i,k}}
since $x_k \in \Delta^n$ and $e^T s = 0$. Hence, for this $s$, the objective
\disp{g_k^T s = \sum_{j=1}^n g_{j,k} s_j = - \sum_{j=1}^n g_{j,k} x_{k,j} + g_{i,k}}
using \req{defs}. The smallest value of $g_k^T s$ over all such vertices
then occurs for the index (perhaps indices) for which $g_{i,k}$ is smallest,
and we use such an $i$ with \req{defs} to define $\sck$.

Thirdly, notice that Steps 3 and 4 are optional; the algorithm is
guaranteed to converge without them, but almost always moving beyond the
Cauchy point improves performance. From a theoretical perspective,
the Cauchy point plays two roles, it guarantees convergence and in
non-degenerate cases it also ultimately predicts the optimal face on
the unit simplex. Once this happens, the exact solution
$\xck + s_k$ from \req{eslls} will also be that of \req{slls},
and $\alpha_k = 1$ will result from Step 4. Steps 3 and 4 might be skipped if,
for example, the set $\calA(\xck)$ differs significantly from its
immediate predecessor.

Finally, the equality-constrained linear least-squares problem \req{eslls}
may be written more compactly as
\eqn{ceslls}{\minimizeover{s_{\calI_k}} \half
 \| A_{\calI_k} s_{\calI_k} - ( b - A \xck) \|^2
\; \mbox{subject to} \; e^T s_{\calI_k} = 0}
to determine the ``free'' components $s_{\calI_k}$ of $s$, where
$\calI_k = \calN \setminus \calA(\xck)$. That is, it only involves
the ``free'' columns $ A_{\calI_k}$ of $A$. To solve a generic singly-constrained
linear least-squares problem
\eqn{sells}{\minimizeover{s \in \smallRe^n} m(s) \define
 \half \| A s - b \|^2 \; \mbox{subject to}\; e^T s = 0}
at least two possibilities are available.

The first is to note that the solution
to \req{sells} must satisfy the optimality conditions
\disp{A^T ( A s - b ) + \lambda e = 0 \; \mbox{and} \; e^T s = 0}
for some scalar Lagrange multiplier $\lambda$. This then leads to the
linear system
\disp{ \mat{cc}{A^T A & e \\ e^T & 0 } \vect{s \\ \lambda} = \vect{ A^T b \\ 0}}
or, on defining $r = b - A s$, the expanded system
\disp{ \mat{ccc}{I & A & 0 \\ A^T & 0 & e  \\ 0 & e^T & 0 }
 \vect{r \\ s \\ \lambda} = \vect{ b \\ 0 \\ 0}.}
or even its solution via
\disp{\vect{ r \\ s } = \vect{ p \\ q } - \lambda \vect{ u \\ v },
\;\mbox{where} \mat{cc}{I & A \\ A^T & 0} \mat{cc}{p & u \\ q & v }
  = \mat{cc}{ b & 0 \\ 0 & e } \;\mbox{and}\; \lambda = \frac{q^T e}{v^T e}.}
Since we are primarily only interested in the $s$ component of the
solution, the latter may be rewritten as
\eqn{nells}{s = q - \lambda v, \;\mbox{where}\; A^T A q = A^T b, \;
 A^T A v = - e \;\mbox{and}\; \lambda = \frac{q^T e}{v^T e}.}
Each of the above systems may be solved using symmetric, indefinite
factorization, and in the special case \req{nells}, Cholesky factorization
of $A^T A$ suffices.

A second possibility is to solve---possibly approximately---the problem
using an iterative method. Standard techniques such as the preconditioned
conjugate-gradient normal equation (CGNE) method may be applied so long
as the preconditioning step respects the single linear constraint. That is,
given a gradient $g_k$ of $m(s)$ at CGNE iterate $s_k$, the required slope
$g_k^T v_k$ is computed to satisfy
\disp{ \mat{cc}{ P & e \\ e^T & 0 } \vect{v_k \\ \mu_k} = \vect{g_k \\ 0}}
for some suitable symmetric approximation $P$ to $A^T A$; this is a
special case of the projected conjugate gradient method
\cite{GoulHribNoce01:sisc} applied to \req{sells}.

\subsection{Projections onto the unit simplex and projection arcs}

\subsubsection{Projections onto the unit simplex}

Given the vector $v$, we let $v_{(i)}$, $i = 1,\ldots,n$,
be the value of its $i$-th largest component.
The basic $O(n\log n)$ method for finding the projection $P(v)$ is due to
\cite{HeldWolfCrow74:mp}, and has been improved over the
years, see \cite{Cond16:mp,DaiChen22,BergFrie08:sisc} and the citations within.
A typically efficient method is as follows:

%% \algo{alg-proj-basic}{projection onto the unit simplex
%% \cite{HeldWolfCrow74:mp}}{
%% \begin{enumerate}
%% \item Sort $v$ so that $v_{(1)} \geq v_{(2)} \geq \ldots \geq v_{(n)}$.

%% \item Define
%% \disp{j_{\max} \define \max_{1 \leq j \leq n}
%%  \left\{ j \st \left( \bigsum_{i=1}^j v_{(i)} - 1\right) / j < v_{(j)} \right\}.}

%% \item Define
%% \disp{\tau \define \left(\bigsum_{j=1}^{j_{\max}} v_{(j)} - 1\right)/j_{\max}.}

%% \item Set $P(v)_i \define \max [v_i - \tau ,0]$ for $i = 1, \ldots ,n$.
%% \end{enumerate}
%% }

%% The quicksort algorithm is frequently recommended in step 1 of Algorithm 1.
%% However, all that is actually needed is the largest entry of the
%% unsorted set of components, and for this the heapsort algorithm is
%% often preferred.

\algo{alg-proj-heap}{projection onto the unit simplex via heapsort
\cite{BergFrie08:sisc}}{
\begin{enumerate}
\item Build a heap $\calH$ with largest entry $h$ from the components of $v$.
\item For j = 1,\ldots,n, do

\begin{enumerate}
\item Set $v_{(j)} \define h$.

\item If \disp{\left(\bigsum_{i=1}^j v_{(i)} - 1\right)/j \geq v_{(j)},}
exit the loop.

\item Set $j_{\max} \define j$.

\item Remove $h$ from $\calH$, and restore the heap $\calH$ with new
largest entry $h$
\end{enumerate}

\item Define
\disp{\tau \define \left(\bigsum_{j=1}^{j_{\max}} v_{(j)} - 1\right)/j_{\max}.}

\item Set $P(v)_i \define \max [v_i - \tau ,0]$ for $i = 1, \ldots ,n$.
\end{enumerate}
}

\noindent
Such a method is based on satisfying the optimality conditions for the
projection problem
\disp{P(v) = \argminover{p \in \Delta^n} \half \| p - v\|^2}
by the constructive finite iteration in Step 2.

\subsubsection{Projection arcs}

We next consider the arc $P(x+\alpha d)$ from a given $x \in \Delta^n$ in
the generic direction $d$ for positive $\alpha$. Since $x$ lies within
$\Delta^n$,  the arc will be piecewise linear as $\alpha$ increases,
and will change direction whenever
one (or more) of its components shrinks to zero (and is fixed thereafter).
We refer to the points at which the direction changes as {\em breakpoints},
and the vectors that join adjacent breakpoints as {\em segments}.

Suppose that $v$ is such a breakpoint, and that $\calA \define \calA(v)$
is as in \req{active-set}. Then the next segment $s$ is in the direction
\disp{\argmin{s \in \Re^n} \half\|s-d\|^2 \;\mbox{subject to} \; e^T s = 0
\, \mbox{and}\; s_{\calA} = 0,}
for which the necessary optimality conditions
\disp{ s_{\calI} - d_{\calI} + \lambda e_{\calI} = 0,\;  e_{\calI}^T s_{\calI} = 0,
\, \mbox{and}\; s_{\calA} = 0}
give that $s$ has components
\eqn{s}{s_{\calI} = d_{\calI} - \frac{e_{\calI}^T d_{\calI}}{e_{\calI}^T e_{\calI}}e_{\calI}
 \;\tim{and}\; s_{\calA} = 0}
where $\calI \define \calN \setminus \calA$.

Now suppose that at the next breakpoint $v^+$ variable $i$ reaches zero,
and thus that $\calI^+ = \calI \setminus \{i\}$. It then follows from \req{s}
that the new segment $s^+$ emanating from $x^+$ satisfies
\eqn{splus}{s^+_{\calI^+} = d_{\calI^+}
 - \frac{e_{\calI^+}^T d_{\calI^+}}{e_{\calI^+}^T e_{\calI^+}}e_{\calI^+}
 \;\tim{and}\; s_{\calA^+} = 0.}
In particular, if $j \in \calI^+ \subset \calI$,  it follows from
\req{s} and \req{splus} that
\disp{s_j = d_j - \frac{e_{\calI}^T d_{\calI}}{e_{\calI}^T e_{\calI}} \;\mbox{and}\;
 s^+_j = d_j - \frac{e_{\calI^+}^T d_{\calI^+}}{e_{\calI^+}^T e_{\calI^+}}}
and particularly that
\eqn{diffs}{s^+_j = s_j + \gamma_i, \;\mbox{where}\;
\gamma_i = \frac{e_{\calI}^T d_{\calI}}{e_{\calI}^T e_{\calI}}
 - \frac{e_{\calI^+}^T d_{\calI^+}}{e_{\calI^+}^T e_{\calI^+}}
= \frac{e_{\calI}^T d_{\calI}}{ |\calI|}
 - \frac{e_{\calI}^T d_{\calI} - d_i}{ |\calI|- 1}
= \frac{|\calI| d_i - e_{\calI}^T d_{\calI}}{|\calI|(|\calI|- 1)}
}
since $e_{\calI^+}^T e_{\calI^+} = |\calI^+| = |\calI|- 1 = e_{\calI}^T e_{\calI} -1$
and
$e_{\calI^+}^T d_{\calI^+} = e_{\calI}^T d_{\calI} - d_i$. Thus each nonzero
component of the new segment is that of the old segment shifted by
{\em the same} $\gamma_i$. We may also use the relationships
$e_{\calI}^T s_{\calI} = 0 = e_{\calI^+}^T s^+_{\calI^+}$ together with \req{diffs} to
deduce that
\disp{0 = \sum_{j \in \calI^+} s^+_j = \sum_{j \in \calI^+} ( s_j  + \gamma_i ) =
\sum_{j \in \calI} s_j  - s_i + \gamma_i |\calI^+| =
- s_i + \gamma_i |\calI^+|
}
and thus we have the alternative definition
\disp{\gamma_i  = \frac{s_i}{|\calI^+|}.}
Since $v_i + \alpha s_i = 0$, and $v_i > 0$, and $\alpha > 0$, we must have
that $s_i < 0$ and consequently $\gamma_i < 0$. Hence the nonzero components
of successive segments decrease; those that start negative stay so, while
those that are initially positive move towards negativity.

We calculate breakpoints by examining the segment $s$ emanating from $v$
and picking $v^+ = v + t s$, where
\disp{\tmin = \min_{s_j<0} t_j, \; \mbox{where}\; t_j = - v_j/s_j.}
By definition of $i$, we have $s_i < 0$ and $\tmin = t_i = - v_i/s_i$.
Now suppose that $s_j < 0$ for component $j \neq i$. Then, as in the
previous paragraph, $s_j^+ = s_j + \gamma_i$, and hence $s_j^+ < 0$.
Moreover
\disp{t_j^+ - t_j = - \frac{v_j + ts_j}{s_j + \gamma_i} + \frac{v_j}{s_j}
= \frac{v_j s_i^2 / |\calI^+| + v_i s_j^2}{ s_j^+ s_i s_j} < 0}
since the numerator is positive and the denominator negative,
and hence $t_j^+ < t_j$. It might be tempting to hope that if
$\tmin \leq t_j < t_k$ for some pair $j,k \neq i$ then $t^+_j < t^+_k$.
For if this were so, it would be possible to sort the indices at the
beginning of the arc, and thereafter pick breakpoints sequentially
according to their sorted order;
this is one of the keys to efficiency in similar methods for the
simpler box-shaped feasible domain $l \leq x \leq u$
\cite[\S17.3]{ConnGoulToin00}.
Unfortunately, this is not the case as the following example shows.

Consider the value $v = (0.01, 0.01, 0.05, 0.93) \in \Delta^4$, and the
direction $s = d = (-10,-1,-10,21)$ for which $e^T s = 0$. Then
$\tmin = t_1 = 0.001 < t_3 = 0.005 <  t_2 = 0.01$, and hence $t = 0.001$, giving
$v^+ = (0, 0.009,0.04,0.951)$,
$s^+ = (0. -4.3333, -13.3333, 17.6667)$ and
$\tmin^+ = t^+_2 = 0.0020769 < t^+_3 = 0.003$. Thus it seems likely that it is
necessary to sort indices at each breakpoint when decoding where the
next segment ends---in practice, we have observed that it is
common for the order to be retained, and so the lack of a guarantee is
regrettable.


%\disp{\arr{rl}{t_j^+ - t_k^+ \ngap & =
% \bigfrac{v_k + t s_k}{s_k + \gamma_i} - \bigfrac{v_j + t s_j}{s_j +\gamma_i}\\
% & = \bigfrac{(t-t_k) s_k}{s_k + \gamma_i}
%      - \bigfrac{(t-t_j) s_j}{s_j + \gamma_i} \\
%}}
%by definition of $t_j$ and $t_k$, the bound $t_j - t < t_k - t$

\subsubsection{Objective function evolution along a projection arc}

Our next goal is to consider the evolution of the objective function
\disp{f(P(x+\alpha d)) = \half \| A P(x+\alpha d) -b\|^2} as $\alpha$
increases from 0 from $x \in \Delta^n$ in the segment direction $d$.
As we have already seen, we may break the arc $P(x+td)$ into pieces,
as we shall consider $v+t s$ from the breakpoint $v$ along the segment
$s$ for $t \geq 0$. Thus, we are concerned with the convex quadratic
\disp{f(v+ts) = \half \|r\|^2 + t f^{\prime} + \half t^2 f^{\prime\prime},
 \;\mbox{where}\; r \define A v - b, \; p \define A s, \;
f^{\prime} \define r^T p \;\mbox{and}\;
 f^{\prime\prime} \define \|p\|^2.}
There are clearly three possibilities. It might be that $v$ is a (local)
minimizer of $f$ along the arc; this will happen if $f^{\prime} \geq 0$.
If this is not the case, either we may move to the end of the segment
as $f(v+ts)$ continues to decrease, or we encounter a local minimizer
en route. The latter occurs when
\disp{t_{\mbox{\scriptsize opt}} \define - f^{\prime} / f^{\prime\prime} \leq \tmin.}
Naively, one might simply evaluate these three possibilities by computing
the product $p = As$ required by $f^{\prime}$ and $f^{\prime\prime}$. However, it is
more efficient to recur the required quantities as the arc search proceeds.

Suppose that the search continues to the next breakpoint, $v^+$, and that
variable $i$ is reduced to zero, i.e., $\calI^+ = \calI \setminus \{i\}$.
Recall from \req{s} that
\disp{p = A s = A_{\calI} s_{\calI} = A_{\calI^+} s_{\calI^+} + a_i s_i,}
where $a_i$ is the $i$-th column of $A$,
and from \req{diffs} that
\disp{p^+ = A s^+ = A_{\calI^+} s^+_{\calI^+}
 = A_{\calI^+} ( s_{\calI^+} + \gamma_i e_{\calI^+}),}
and hence that
\disp{p^+ = p + \gamma_i A_{\calI^+} e_{\calI^+} - s_i a_i.}
But if $w \define  A_{\calI} e_{\calI}$, we may recur
\eqn{w}{w^+ = A_{\calI^+} e_{\calI^+} = A_{\calI} e_{\calI} - a_i = w - a_i}
and subsequently
\eqn{p}{p^+ = p + \gamma_i w^+ - s_i a_i.}
Crucially, both recurrences simply need access to a single---possibly
 sparse---column of $A$. In addition, we may recur
\eqn{r}{r^+ = A v^+ - b = A ( v + \tmin s ) - b = r + \tmin p}
and calculate
\eqn{primes}{\half \|r^+\|^2 = \half \|r\|^2 + \tmin  f^{\prime} +
 \half \tmin^2 f^{\prime\prime}, \;
f^{\prime +} = r^{+ T} p^+ \;\mbox{and}\;
 f^{\prime\prime +} = \|p^+\|^2.}
The main work involved are the two potentially-sparse and two dense vector
additions in \req{w}--\req{r}, and two inner products in \req{primes},
and avoids a more costly matrix-vector product $As$. This is less efficient
than projected search when the feasible region is a box, but this appears
to be inevitable because of the complicating constraint $e^T x = 1$.

Another possibility is to aim for a value along the arc that provides
sufficient decrease of $f(P(x+\alpha d))$ to ensure convergence rather than
for the minimizer.
Such a possibility has been examined in more general contexts using
Goldstein-like conditions \cite[\S12.2]{ConnGoulToin00}, but we prefer a
slightly simpler Armijo version such as that in \cite{Sart93b}. Essentially,
given an initial estimate $\alpha_0$ of the optimal step length $\alpha$,
$f(P(x+\alpha d))$ is computed for a sequence of decreasing values
$\alpha_k = \alpha_0 \beta^k$, $k \geq 0$ with $\beta \in (0,1)$,
and the search terminated at the first $k$ for which
\eqn{suffdec}{f(P(x+\alpha_k d)) \leq f(x) + \eta
 (\nabla_x f(x) )^T ( P(x+\alpha_k d) - x )}
for given (typically small) $\eta \in (0,1)$.
If we write $s_k = P(x+\alpha_k d) - x$ and $r_k = A P(x+\alpha_k d) -b$,
then $r_k = r + A s_k$, where we recall that $r = A x - b$.
Hence we may rewrite \req{suffdec} as
\disp{\frac{1}{2} r_k^T r_k \leq \frac{1}{2} r^T r + \eta (r^T r_k - r^T r ).}
Thus each $k$ requires the computation of a projection $P(x+\alpha_k d)$,
a matrix-vector product $As_k$ and a pair of new inner products
$r^T r_k$ and $r_k^T r_k$. How this compares to the earlier, exact method
depends entirely on the choice of $\alpha_0$, and a good choice isn't obvious.
Indeed, a large initial $t_0$ may result in a wasteful sequence of identical
evaluation points $P(x+\alpha_k d)$.


%Questions still to answer:
%\begin{enumerate}
%\item is it possible to compute the next breakpoint without another complete
%ordering of $-v_i/s_i$ (= distances to boundary of simplex)? False.

%\item can we compute $f(P(x+\alpha v))$ efficiently? Without recomputing
%everything at each breakpoint as the arc direction changes?

%\end{enumerate}
%I suspect that the special update \req{diffs} may be crucial here for both
%questions.

\bibliographystyle{plain}
%\input{/home/nimg/pCloudDrive/papers/refs/refs}
\bibliography{g24}

\end{document}
